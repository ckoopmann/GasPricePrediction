\chapter{Conclusions}\label{Sec:Conc}
In the following paragraphs the above illustrated results will be used to answer the research questions posed in the introduction. Afterwards a few additional observations and learnings obtained during the work on this thesis will be summarised and an outlook on future possible extensions of this work will be given.
\section{Research Questions}
\subsection{Performance of recurrent neural networks}
When assessing the performance of recurrent neural networks as a forecasting tool the results of the two prediction problems differ quite substantially. In the case of the price level prediction the simple lagged value approach using today's price as prediction for tomorrow outperforms both recurrent and feed forward neural networks. The best performing data driven models in this context are in fact those, whose predictions most closely resemble the lagged value. This lack of a predictive advantage might be explained by two different factors. One possible explanation is the assumption that the highly liquid natural gas market represents a very efficient pricing mechanism whose current price already takes all available information into account. When looking at the literature reviewed above regarding price forecasting of natural gas futures, one does in fact see that the results are very mixed so far. Even many of the papers reporting positive performance of machine learning algorithms only compare their models to very similar approaches to evaluate the advantage of a particular feature such as wavelet decomposition or feature selection. Many of these papers do not include a comparison of the model against the price expectations of the market.

The alternative explanation, might be a lack of data. Since the data is of only daily granularity and the model training was limited to complete observations, the models were only trained on around 1000 training observations. Especially recurrent neural networks and particularly LSTM models contain a very large number of trainable parameters. In this case the parameters some times outnumbered the available training observations by a factor of four to one. This makes efficient training of these models while avoiding overfitting particularly hard and might also be an explanation of the relatively bad performance of the multivariate models. 

With respect to the binary prediction problem the evaluation yields a much more positive conclusion. One the one hand this might be due to the relatively weaker base line reference model which is based on an equal distribution of the probability of being minimal across the remaining days. Unlike the reference model in the price level prediction this forecast does not contain any information of the current market expectations. This reference is in fact outperformed by all but two of the eight different data driven models. However even among these models the performance of recurrent neural networks compares relatively well with three of the four top ranked models regarding cross entropy being of recurrent structure. 
Looking at the monetary evaluation of the derived trading strategies this picture is confirmed with the top three models all being recurrent.
\subsection{Relative performance of LSTM}
Comparing the relative performance of the LSTM model with that of the simpler RNN architecture the results are more consistent across the two prediction problems. When looking at the ranking of the models according to the loss function within the uni- and multivariate groups, the LSTM model outperforms the simple RNN in both prediction problems. The picture is slightly different in the monetary evaluation where the univariate simple RNN model slightly outperforms the LSTM, however it should be noted that the model was not optimized on this metric and the ranking slightly changes according to the choice of the $t_0$ parameter. Across all prediction problems and evaluation metrics the univariate LSTM model shows a consistently good performance being the best or second best model in all rankings. 
\subsection{Business value}
The potential business value that the recurrent neural networks developed above can offer as a tool to support the purchasing of natural gas mostly derives from the results of the binary prediction problem. The good performance of the LSTM regarding binary cross entropy suggest that the predictions when interpreted as probability estimates already provide a useful piece of information for human decision making. However the most direct and tangible value is displayed by the good performance of the simple trading strategy derived from these predictions and the savings that can be realised compared to the equal distribution reference. While the performance of a human trader might be a more competitive reference, however it should be remembered that these models could run automatically and thereby realise further savings to the wage bill .
\section{Other observations}
Beyond the findings presented above additional observations conducting the work presented in this thesis.
\section{Outlook}
