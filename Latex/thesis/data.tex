\chapter{Data}\label{Sec:Data}
The data used for this analysis is downloaded from the \textit{Thomson Reuters Eikon} database. For this purpose  a Python script was implemented that downloads all necessary time series through the \textit{Eikon Web and Scripting APIs}. The following paragraphs give an overview over the structure and content of the data both on the target variable as well as the input variables used for the model. 
\section{Target Variable}
\subsection{Price Level Prediction}
The target variable to be predicted in this prediction problem is the closing price for the TTF gas future with delivery in the next month, referred to as front month. An advantage of this specification of the target variable is, that one gets one continuous time series for which data can be downloaded for a long period of time. However this definition also poses a significant challenge. The challenge lies in the fact that the product, which is traded, effectively changes at the change of the month as the delivery period changes. For example the Front Month price quoted on September 30 refers to natural gas delivered in the month of October, whereas on October 1 it refers to delivery in November. Due to the strong seasonality in the natural gas price this can lead to structural breaks and large price changes at the end of each month. As an illustration of this problem both the front month price as well as the price of each months future as traded before it becomes the Front are plotted for the second half of 2016 in figure \ref{fig:ttf_monthly_level_2016_02}.

\begin{figure}[h!]
  \centering
\includegraphics[width=0.8\textwidth,keepaspectratio]{\string"../../Plots/ttf_monthly_level_2016_02\string".png}
  \caption{TTF Front Month vs. Individual Months}\label{fig:ttf_monthly_level_2016_02}
\end{figure}

As one can see the price of the front month jumps at the end of each month to the level at which the next month was traded previously.  This means, that while in this thesis the front month is referred to as one continuous target variable, it might also be viewed as a collection of separate time series. To ensure adequate performance of the models, this idea is taken into consideration in the implementation of the model training. That means that, when generating predictions for observations at the beginning of each month past price values of the monthly future for this delivery period are used as input variables. This is done instead of choosing the price values of what was the front month before to prevent the structural breaks a the turn of the month from affecting model performance.

To get a first overview of the inter temporal dependencies across time in the target variable as well as a hint on the performance of the autoregressive reference models both the autocorrelations  as well as partial autocorrelations of the TTF front month are plotted in figures \ref{fig:acf_ttf_fm_level} and \ref{fig:pacf_ttf_fm_level}. The very high values of all auto correlations as well as the near zero partial autocorrelations for lags above one day suggest modelling the target variable using an AR(1) model, which will probably result in a parameter estimate very close to one. 



\begin{figure}[h!]
  \centering
\includegraphics[width=0.8\textwidth,keepaspectratio]{\string"../../Plots/acf_ttf_fm_level\string".png}
  \caption{Auto Correlations TTF Front Month Closing Prices}\label{fig:acf_ttf_fm_level}
\end{figure}

\begin{figure}[h!]
  \centering
\includegraphics[width=0.8\textwidth,keepaspectratio]{\string"../../Plots/pacf_ttf_fm_level\string".png}
  \caption{Partial Auto Correlations TTF Front Month Closing Prices}\label{fig:pacf_ttf_fm_level}
\end{figure}

\FloatBarrier
\subsection{Binary Prediction}
The target variable of the binary prediction problem can best be described as the event: "Today's front month closing price is minimal among all remaining trading days of this month." (True = 1, False = 0). This target variable can be generated from the price level data in the following way, where the index $d$ denotes the trading day within each month and the index $M$ the month. $I$ is the indicator function return one or zero depending on weather the statement in the argument is true or false:
\begin{align*}
y_{d,M}^{binary} = \prod_{i >= d} I(y_{d,M}^{level} \leq y_{i,M}^{level})
\end{align*}
To avoid misunderstanding it is important to note that this variable does not correspond to the question of whether a certain trading day is minimal among all days of that month. In fact there is no theoretical limit on the amount of positive observations of this variable. To illustrate this figure 
In case of constantly rising prices all observations of this variable would be positive.
\ref{fig:binary_plot_many_positive} shows the binary target variable and the front month price level for a month with a lot of positive observations.
\begin{figure}[h!]
  \centering
\includegraphics[width=0.8\textwidth,keepaspectratio]{\string"../../Plots/binary_plot_many_positive\string".png}
  \caption{Binary target variable and price level for December 2016}\label{fig:binary_plot_many_positive}
\end{figure}
For the whole data around a quarter of the observations are positive, however this share can vary significantly as shown in table \ref{tab:balance-overview}. It is important to note that the value of this variable at any point in time contains information regarding future prices that is not available at that time point. Therefore the value of that variable for each trading day can only be observed after the end of the month and not on the trading day itself. For this reason values of this target variable are never used as an input variable and the univariate models in the binary prediction case also take the past values of the price level as inputs.

\section{Input Variables}\label{Sec:Input}
The following paragraphs will briefly describe all potential input variables that were included in the variable selection. These variables can be sorted into three categories: Energy commodity prices, gas market fundamentals and exchange rates, each of which will be illustrated separately. The initial choice of candidate variables was based on input from energy traders at BASF SE. The development of each input variable across time is plotted in figures  \ref{fig:TTFDAlevel} - \ref{fig:EURGBPFXlevel}.
\subsection{Energy Commodity Prices}
The variables conceptually most closely related to the target variable are surely prices of other energy commodities. As argued in  section \ref{Sec:Literature} natural gas shares characteristics with both oil and electricity markets. Therefore it is somewhat natural to include prices for these commodities in addition to natural gas prices for other hubs and delivery periods.
\begin{description}
\item[TTF Day Ahead /Spot:] At the TTF Day Ahead (TTFDA) market, which is sometimes also referred to as the spot market, natural gas is traded every day for delivery during the next day. This market is one of the most liquid markets at any gas trading hub and enables market participants to react to daily fluctuations in both demand and supply.
\item[NBP Front Month:] The front month price at the National Balancing Point is the equivalent of the target variable for the UK market. As explained above this is the oldest, most established and most liquid virtual trading point in Europe. Unlike the TTF, which is traded in Euro, this product is traded in British Pound and therefore the relative price difference between these markets can be influenced by the exchange rate.
\item[Oil Front Month:] This time series describes the price of  Brent crude oil with contract conditions equivalent to that of the target variable. This product is priced in US Dollars.
\item[Electricity Base Load Front Month:] This is the price of the so called \textit{Phelix} Base Future for delivery of electricity the next month from 0:00 - 24:00.
\item[Electricity Peak Load Front Month:] This price is equivalent to the base load price with the difference that the delivery is limited to the peak hours 09:00 - 20:00. Both base and peak load futures are traded in Euros.
\end{description}
All energy commodity prices are represented by daily closing prices.
\subsection{Gas Market Fundamentals}
The second group of input variables contain data on the four main fundamental factors presumed to drive the natural gas market: consumption, production, storage and pipeline flows. 
\begin{description}
\item[Consumption:] The consumption can be divided into two types. The first type describes the amount of natural gas consumed through Local Distribution Zones (LDZ), this is the channel through which residential and the vast majority of industrial users receive their gas. The Non-LDZ consumption describes the demand of users directly connected to the high pressure gas grid, which are mostly power stations and very large industrial consumers. Both demands are used as separate input variables on a daily resolution for the Dutch market.
\item[Production:] Daily output of British gas wells on the Continental Shelf as well as daily output of all Dutch fields are aggregated into one variable each and used as separate input variables.
\item[Storage:] Natural gas storage levels in the Netherlands and UK are each used as a separate variable. These storage levels are highly seasonal since depots are usually filled during the summer months and then drained to fulfil the higher demand in the heating season.
\item[Pipeline Flows:] The \textit{Interconnector} (IUK) and \textit{Balgzand Bacton Line} (BBL) are the two main natural gas pipelines connecting the British and European markets. Net gas flows on both pipelines are additional sources of demand or supply on the European market and are therefore included as additional variables.
\end{description}
\subsection{Exchange Rates}
As mentioned above, some of the commodity prices included as inputs are denoted in US Dollar or British Pound. Therefore the exchange rates of both currencies to the Euro are included as daily closing prices to enable the models to react to the effects of foreign exchange rate volatility.
\subsection{Data Availability}\label{Sec:Data-Availability}
In table \ref{tab:data-availability} one can see that the different categories of input variables vary strongly regarding the amount of data available. Whereas the commodity prices and exchange rates are available starting in 2010, the gas market fundamentals data starts between 2012-2014 with many variable providing around half as many non missing observations as the target variable. When training, tuning and evaluating models the data will be limited to complete observations where none of these variables is missing. This significantly reduces the amount of data available to each model and poses challenges to models with a lot of trainable parameters.

\input{\string"../tables/data_availability\string".tex}