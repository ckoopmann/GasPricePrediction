\section{Data}\label{Sec:Data}
The data used for this analysis is downloaded from the \textit{Thomson Reuters Eikon} database. For this purpose I have implemented a Python script that downloads all necessary time series through the \textit{Eikon Web and Scripting APIs}. In the following I will give an overview over the structure and content of the data both on the target variable as well as the input variables used for the model. 
\subsection{Target Variable}
\subsubsection{Data Structure}
The target variable to be predicted in this thesis is the closing price for the TTF gas future with delivery in the next month, referred to as front month. An advantage of this specification of the target variable is, that one gets one continuous time series for which data can be downloaded for a long period of time. However this definition also poses a significant challenge. The challenge lies in the fact that the product which is traded effectively changes at the change of the month as the delivery period changes. For example the Front Month price quoted on September 30 refers to natural gas delivered in the month of October, whereas on October 1 it refers to delivery in November. This leads to structural breaks and large price changes at the end of each month. To illustrate this I have plotted both the front month price as well as the price of each months future as traded before it becomes the Front Month for the second half of 2016 in Figure \ref{fig:ttf_monthly_level_2016_02}. As one can see the price of the front month jumps at the end of each month to the level at which the next month was traded previously.  This means, that while in this thesis I refer to the front month as one continous target variable, it might also be viewed as a collection of separate time series. To ensure adequate performance of the models this idea is implemented in the implementation of the model training. That means that, when generating predictions for observations at the beginning of each month past price values of the monthly future that is the current front month are used as inputs. This is done instead of choosing the price values of what was the front month before to prevent the structural breaks a the turn of the month to affect model performance.  

To get a first overview of the inter temporal dependencies across time in the target variable as well as a hint on the performance of the autoregressive reference models both the autocorrelations  as well as partial autocorrelations of the TTF front month are plotted in figures \ref{fig:acf_ttf_fm_level} and \ref{fig:pacf_ttf_fm_level}. The very high values of all auto correlations as well as the near zero partial autocorrelations for lags above 1 day suggest modelling the target variable using an AR(1) model, which will probably result in a parameter estimate very close to 1. 

\begin{figure}
  \centering
\includegraphics[width=0.8\textwidth,keepaspectratio]{\string"../../Plots/ttf_monthly_level_2016_02\string".png}
  \caption{TTF Front Month vs. Individual Months}\label{fig:ttf_monthly_level_2016_02}
\end{figure}



\begin{figure}
  \centering
\includegraphics[width=0.8\textwidth,keepaspectratio]{\string"../../Plots/ttf_fm_level\string".png}
  \caption{TTF Front Month Closing Prices}\label{fig:ttf_fm_level}
\end{figure}


\begin{figure}
  \centering
\includegraphics[width=0.8\textwidth,keepaspectratio]{\string"../../Plots/acf_ttf_fm_level\string".png}
  \caption{Auto Correlations TTF Front Month Closing Prices}\label{fig:acf_ttf_fm_level}
\end{figure}

\begin{figure}
  \centering
\includegraphics[width=0.8\textwidth,keepaspectratio]{\string"../../Plots/pacf_ttf_fm_level\string".png}
  \caption{Partial Auto Correlations TTF Front Month Closing Prices}\label{fig:pacf_ttf_fm_level}
\end{figure}

\subsection{Input Variables}\label{Sec:Input}
\subsubsection{Other Energy Commodities}
\begin{description}
\item[TTF Day Ahead /Spot:]
\item[NBP Front Month:]
\item[Oil Front Month:]
\item[Electricity Base Load Front Month:]
\item[Electricity Peak Load Front Month:]
\end{description}
\subsubsection{Gas Market Fundamentals}
\begin{description}
\item[LDZ Gas Consumption:]
\item[Non-LDZ Gas Consumption:]
\item[Production:]
\item[Storage:]
\end{description}
\subsubsection{Exchange Rates}
\begin{description}
\item[Euro - US Dollar:]
\item[Euro - Pound:]
\end{description}
\subsubsection{Correlation Analysis}
\subsubsection{Data Availability}