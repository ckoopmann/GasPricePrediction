\chapter{The natural gas market}
In this section the reader will be given a brief overview of the history and current situation in the European natural gas market. Special emphasis will lie on the different ways natural gas is traded and how the current system  evolved. 

\section{History}
The first commercially used gas was in fact produced from coal and used as a lighting fuel beginning in the late 18th century in Britain, afterwards spreading to the United States and Continental Europe during the 19th century. The first well extracting natural gas directly from the ground was dug 1821 in Fredonia, New York (\cite{heather_evolution_2015}). At the beginning this gas could only be consumed at locations close to its source. The expansion of the application of natural gas into fields such as heating, chemical processing and power generation was connected with the construction of effective long distance pipelines. In Europe the advent of large scale natural gas imports from the former Soviet Union and North Africa in the mid 20th century required especially large infrastructure investments. To recover this investment long term contracts with durations of 25 to 30 years were negotiated by the usually state owned entities that controlled the natural gas trade. These entities usually integrated many different areas of the value chain from exploration over pipeline transport to the sale of natural gas. During this time there were relatively few participants in the European natural gas market that often occupied a monopoly position in their respective national markets.
For the sale of gas from the then newly developed Groningen field the Dutch producers developed an oil based pricing system, where the price for long term natural gas contracts was tied to the market price for crude oil. This pricing system spread to many other natural gas contracts and became the predominant pricing system in the European market.
An alternative pricing regime developed in the 1970s in the United States. This pricing regime is based on defining regional gas trading hubs and developing independent markets for gas traded at each of these hubs. In the US the Henry Hub in Louisiana has evolved as the main trading hub. Hub pricing spread from the US to the UK with the privatisation of the British gas industry in the 1980s and the establishment of the National Balancing Point (NBP) as the main trading hub for the UK market. Whereas the Henry Hub evolved from a physical hub connecting various long distance pipelines the NBP was established as a Virtual Trading Point (VTP) with the aim of developing a new price reference for natural gas trading. Regulatory reform and privatisation of the gas industry was a key requirement for the establishment of these markets since it  allowed for the separation of distribution networks from natural gas trading activities and thereby ensured that all market participants had equal access to the distribution infrastructure. Aiming to establish an integrated European energy market these regulatory changes have later been adopted by the European Union and spread to the other member countries. This has lead to the establishment of Continental European virtual trading points with the Dutch Title Transfer Facility (TTF) emerging as the main hub. The establishment of these hubs has allowed natural gas to evolve into a commodity traded on spot and future markets similar to crude oil. With the improvement of the technology and infrastructure for trading Liquified Natural Gas a new alternative to pipelines has emerged and the gas market has become more globally integrated. Due to the recent boom in domestic shale gas production the United States has drastically reduced its LNG imports and is becoming a net exporter (\cite{iea_world_2017}). This has lead to an oversupply in LNG markets with a resulting downward pressure on gas prices. As a result of this the long term oil indexed contracts have become a lot less attractive for buyers, who have pushed for switching to hub pricing. The share of oil indexed pricing in the North West European market (including France, Germany, UK and Netherlands) has fallen from 70 percent in 2005 to just over 10 percent in 2014. Accordingly hub pricing accounted for 90 percent in 2014 (\cite{heather_evolution_2015}, p. 14). 
In the past years various new national trading hubs were created throughout Europe with the two German hubs NetConnect Germany (NCG) and Gaspool (GP) chief among them. However these new exchanges still lack far behind the established hubs (NBP, TTF) in terms of liquidity and trading volume.

\begin{figure}[H]
  \centering
\includegraphics[width=0.8\textwidth,keepaspectratio]{\string"../../Plots/hub_map\string".png}
  \caption{Selected European Gas Hubs. Primary Focus: TTF, Secondary Focus: NBP, GP}\label{fig:hub_map}
\end{figure}

\section{Types of natural gas trading}
In the context of the above outlined history of natural gas three basic ways of trading have emerged, which are visualised in Figure \ref{fig:trading_types}.

\paragraph{Bilateral negotiations}
The most classical way of trading natural gas are individually negotiated contracts between supplier and buyer. These are often very long term contracts aimed at ensuring a reliable physical supply of natural gas. Just as the other contract features, pricing systems differ from contract to contract and are not transparent to other market participants. Prices are usually specified to follow a specific benchmark such as the oil price or the gas price at a certain trading hub. 

\paragraph{Over-the-Counter}
Over the counter trading is a specific system of trading various financial assets. Although the contracts in this system are standardised the trading occurs individually between selling and buying party without the supervision of a financial exchange. This results in less transparency since both quotes and closed trades are not necessarily visible to other market participants. Also there is no clearing house involved in the trading which means that the trading parties are exposed to the credit risk of the opposite party defaulting on its obligations. The products available on natural gas OTC markets contain both spot trading as well as forwards and options.

\paragraph{Exchange trading}
Analogously to OTC trading, products traded at financial exchanges are standardised contracts such as futures or options. The key difference lies in the way trades between parties are conducted. At financial exchanges the market participants post sell or buy offers publicly to all other traders who then have an equal chance to trade on that offer. That way all offers and trades are visible to the whole market. Another difference is the existence of a clearing house which takes over the credit risk of both parties and pays out open debts in case of a default. Overall trading at exchanges is more tightly regulated and offers less risk with more transparency than OTC trading. The largest exchanges for trading natural gas derivatives of European hubs are the Intercontinental Exchange (ICE) and the European Energy Exchange (EEX). Both of these exchanges offer a variety of derivatives for TTF and NBP gas.

\begin{figure}[H]
  \centering
\includegraphics[width=0.8\textwidth,keepaspectratio]{\string"../../Plots/trading_types\string".png}
  \caption{Types of Natural Gas Trading}\label{fig:trading_types}
\end{figure}

\section{Natural gas derivatives}
The natural gas contracts relevant for the work done in this thesis can be divided into spot and future contracts. Fundamentally these contracts work in the same way. They cover the delivery of a certain amount of gas at an equal rate over a certain time frame in the future. The difference between these markets lie in the length of the time frame as well as how far the delivery time lies in the future. In the context of this work the delivery of a spot contract lies at most one day in the future and the delivery time frame is at most one day long. This means that the spot market contains basically the day ahead and within day products. Day ahead markets trade gas to be delivered over the next day, whereas within day markets trade hourly products for the remaining hours of the current day. Future markets contain all markets with delivery time frames longer than one day. These markets contain monthly as well as quarterly, seasonal and yearly products. The seasonal products are the winter season from October to March and the summer season. Usually the most intensively traded future of each kind is the one with the nearest delivery time, e.g. the next month or quarter. This future will be called the Front-Month and will be the main focus of this study. 

\begin{table}[H]
	\begin{center}
			\begin{tabular}{c |  c}
				Spot Markets & Future Markets \\
				\hline
		 		Within Day & Months \\
		 		Day Ahead & Quarters \\
		 		 & Seasons (Winter / Summer) \\
		 		 & Calendar Year
			\end{tabular}
	\end{center}
	\caption{Future vs. Spot Markets}
	\label{tab:markets}
\end{table}


