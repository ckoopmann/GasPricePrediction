\section{Introduction}
\subsection{Motivation}
In the recently published World Energy Outlook 2017 the International Energy Agency predicts natural gas to be the fastest growing fuel, accounting for a quarter of all growth in energy consumption until 2040 and overtaking coal to become the second most important fuel. This follows their prediction from 2011 of a "Golden age of Natural Gas" with the growing trade in Liquified Natural Gas and the shale gas revolution of North America being the main growth drivers on the supply side. On the demand side most growth will come from power generation and industrial processing leading to a growing impact of natural gas procurement on the bottom line of many corporations.  At the same time trading natural gas is made more complex by the continuing replacement of long term oil indexed contracts by exchange-traded hub priced contracts. Economic success on the natural gas market will therefore be increasingly dependent on successfully modelling and forecasting the price of these contracts. Despite this growing importance natural gas prices have so far received less attention by the forecasting literature than other commodities or financial markets. This thesis tries to fill that gap by applying state of the art machine learning methods to the prediction of future prices on the European natural gas market. Despite their recent success in other time series prediction problems Long Short Term Memory Networks have not yet been widely applied in this area and therefore constitute a natural candidate for this task. 
\subsection{Topic}
Since this thesis was written in cooperation with an industrial partner in the chemical industry it pursues both a methodological as well as an application oriented objective. On the methodological side the aim is to research the power of Long Short Term Networks in the prediction of financial time series relative to other approaches. Regarding the application the purpose is to develop a tool to serve natural gas gas traders in the chemical industry in the support of daily purchasing decisions. To serve both of these objectives equally well, this thesis analyses two prediction problems:
\begin{description}
\item[Price Level Prediction:] This forecasting problem consists of predicting the closing price of the natural gas future on the next trading day, based on all data available up to the current day.For both the energy commodity as well as wide selection of other financial markets, this or a very similar problem has been researched for a long time. Therefore methods as well as results from this prediction problem could be easily extended and compared to other areas.
\item[Binary Prediction:] Natural gas traders in the chemical industry usually try to purchase natural gas to satisfy a physical demand in the business at an optimal price. This means that they are often limited to only act as a buying party on the market and need to buy a fixed amount of gas within a certain time frame. Under these restrictions the question these market participants have to ask themselves is: "Is today's price minimal, among all remaining trading days until the demand has to be met?". The binary variable corresponding to the answer to this questions is the target in this second prediction problem.
\end{description}
In the interest of both consistency the approach and methods used for both prediction problems will be identical wherever possible.
\subsection{Structure}